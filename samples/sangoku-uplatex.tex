\documentclass[twocolumn]{utarticle}

\usepackage{color}

\newcommand{\hojo}[1]{\textcolor{red}{#1}}
\newcommand{\newjis}[1]{\textcolor{blue}{#1}}

\begin{document}

\section{伝 羅貫中「三國演義」}

\subsection*{第一回~~宴桃園豪傑三結義 斬\newjis{黃}巾英雄首立功}

話説天下大勢、分久必合、合久必分。周末七國分爭、\newjis{幷}\footnote{青字はJIS第3,4水準(JIS X 0213)の文字。}入於秦。及秦
滅之後、楚・漢分爭、又\newjis{幷}入於漢。漢朝自高祖斬白蛇而起義、一統天下。
後來光武中興、傳至獻帝、遂分爲三國。推其致亂之由、殆始於桓・靈二
帝。桓帝禁錮善類、崇信宦官。及桓帝崩、靈帝\newjis{卽}位、大將軍竇武・太傅
陳蕃、共相輔佐。時有宦官曹節等弄權、竇武・陳蕃謀誅之、機事不密、
反爲所害、中涓自此愈\newjis{橫}。

建寧二年四月望日、帝御\newjis{溫德}殿。方陞座、殿角狂風驟起、只見一條大
青蛇、從梁上飛將下來、蟠於椅上。帝驚倒、左右急救入宮、百官\newjis{俱}奔避。
須臾、蛇不見了。忽然大雷大雨、加以冰雹、落到半夜方止、壞却房屋無
數。建寧四年二月、洛陽地震。又海水泛溢、沿海居民、盡被大浪捲入海
中。光和元年、雌\hojo{雞}\footnote{赤字はJIS補助漢字(JIS X 0212)の文字。大部分はJIS第3,4水準(JIS X 0213)にも再度収録されている。}化雄。六月朔、\newjis{黑}氣十餘丈、飛入\newjis{溫德}殿中。秋七月、
有虹見於玉堂、五原山岸、盡皆崩裂。種種不祥、非止一端。帝下詔問羣
臣以災異之由、議\newjis{郞}蔡\hojo{邕}上疏、以爲\hojo{蜺}墮\hojo{雞}化、乃婦寺干政之所致、言頗
切直。帝覽奏歎息、因起更衣。曹節在後竊視、悉宣告左右。遂以他事陷
\hojo{邕}於罪、放歸田里。後張讓・趙忠・封\hojo{諝}\footnote{この字はJIS第3,4水準(JIS X 0213)にも収録されていない。}・段珪・曹節・侯覽・蹇碩・程
曠・夏\hojo{惲}・郭勝十人朋比爲奸、號爲「十常侍」。帝尊信張讓、呼爲「阿
父」、朝政日非、以致天下人心思亂、盜賊蜂起。

時鉅鹿郡有兄弟三人。一名張角、一名張寶、一名張梁。那張角本是個
不第秀才。因入山採藥、遇一老人、碧眼童顏、手執藜杖、喚角至一洞中、
以天書三巻授之曰、『此名《太平要術》。汝得之、當代天宣化、普救世
人。若萌異心、必獲惡報。』角拜問姓名。老人曰、『吾乃南華老仙也。』
言訖、化陣清風而去。

角得此書、曉夜攻習、能呼風喚雨、號爲「太平道人」。中平元年正月
内、疫氣流行、張角散施符水、爲人治病、自稱「大賢良師」。角有徒弟
五百餘人、雲游四方、皆能書符念咒。次後徒衆日多、角乃立三十六方、
大方萬餘人、小方六七千、各立渠帥、稱爲「將軍」。訛言、『蒼天已死、
\newjis{黃}天當立。』又云、『歳在甲子、天下大吉。』令人各以白土、書「甲子」
二字於家中大門上。青・幽・徐・冀・荊・揚・\hojo{兗}・豫八州之人、家家侍
奉大賢良師張角名字。角遣其黨馬元義、暗齎金帛、結交中涓封\hojo{諝}、以爲
内應。角與二弟商議曰、『至難得者、民心也。今民心已順、若不乘勢取
天下、誠爲可惜。』遂一面私造\newjis{黃}旗、約期舉事。一面使弟子唐州、馳書
報封\hojo{諝}。唐州乃逕赴省中告變。帝召大將軍何進調兵擒馬元義、斬之。次
收封\hojo{諝}等一干人下獄。張角聞知事露、星夜舉兵、自稱「天公將軍」、張
寶稱「地公將軍」、張梁稱「人公將軍」。申言於衆曰、『今漢運將終、
大聖人出。汝等皆宜順天從正、以樂太平。』四方百姓、裹\newjis{黃}巾從張角反
者、四五十萬。賊勢浩大、官軍望風而靡。何進奏帝火速降詔、令各處備
禦、討賊立功。一面遣中\newjis{郞}將盧植・皇甫嵩・朱雋、各引精兵、分三路討
之。

且説張角一軍、前犯幽州界分。幽州太守劉焉、乃江夏竟陵人氏、漢魯
恭王之後也。當時聞得賊兵將至、召校尉鄒靖計議。靖曰、『賊兵衆、我
兵寡、明公宜作速招軍應敵。』劉焉然其説、隨\newjis{卽}出榜招募義兵。榜文行
到\hojo{涿}縣、引出\hojo{涿}縣中一個英雄。

\end{document}

このテキストは坂口丈幸さんのページ(http://rtk.art.coocan.jp/)で
フリーで公開されているUTF-8のテキストをJIS X 0213+JIS X 0212の範囲に
変換したものです。

