%
% test of Babel(greek)+UTF8 and CJK multilingual text
%
% This file is originally a test file for Utf82TeX
%%% Utf82TeX sample TeX file for Unix
%%% (c) 2004-2005, Isao YASUDA, isao@yasuda.homeip.net
%%% $Id: utf82tex-sample.tex,v 1.2 2006/12/09 15:04:25 isao Exp $
% and is modified for upLaTeX
%
\documentclass[a4j]{ujarticle}
%%% kcatcode  15: not cjk, 16: kanji, 17: kana, 18: other cjk char
%\kcatcode"E7=15% U+00E7:ç  (Latin-1 Supplement, Latin-1 letters)
%\kcatcode`П=15% U+041F:П (Cyrillic)
\kcatcode`α=15% U+03B1:α (Greek)
\kcatcode`ἀ=15% U+1F00: ἀ (Greek Extended)
\kcatcode`’=15% U+2019: ’ (General Punctuation)

\usepackage{ucs}
\usepackage[utf8x]{inputenc}
\usepackage[10pt]{type1ec}
\usepackage[T1]{fontenc}
\usepackage[polutonikogreek]{babel}
\begin{document}

\selectlanguage{polutonikogreek}
\section{古典ギリシア語 \protect\textlatin{Polytonic Greek}}
%% utf82tex default option
\begin{verse}
Ἄνδρα μοι ἔννεπε, Μοῦσα, πολύτροπον, ὃς μάλα πολλὰ\\
πλάγχθη, ἐπεὶ Τροίης ἱερόν πτολίεθρον ἔπερσε.\\
πολλῶν δ’ ἀνθρώπων ἴδεν ἄστεα καὶ νόον ἔγνω,\\
πολλὰ δ’ ὅ γ’ ἐν πόντῳ πάθεν ἄλγεα ὃν κατὰ θῡμόν,\\
ἀρνύμενος ἥν τε ψῡχὴν καὶ νόστον ἑταίρων.\\
ἀλλ’ οὐδ’ ὧς ἑτάρους ἐρρύσατο, ἱέμενός περ·\\
αὐτῶν γὰρ σφετέρῃσιν ἀτασθαλίῃσιν ὄλοντο,\\
νήπιοι, οἳ κατὰ βοῦς Ὑπερίονος Ἠελίοιο\\
ἤσθιον· αὐτὰρ ὁ τοῖσιν ἀφείλετο νόστιμον ἦμαρ.\\
τῶν ἁμόθεν γε, θεά, θύγατερ Διός, εἰπὲ καὶ ἡμῖν.
\end{verse}

\hfill {\em [ \textit{Ὅμηρος} ]} \qquad\qquad

\vspace{1em}
ΑΒΓΔΕΖΗΘΙΚΛΜΝΞΟΠΡΣΤΥΦΧΨΩ 
αβγδεζηθικλμνξοπρσς τυφχψω\\
ΆἉἊἋἌἍἎἏ ἀἁἂἃἄἅἆἇ ἘἙἚἛἜἝ ἐἑἒἓἔἕ ἨἩἪἫἬἭἮἯ %
ἠἡἢἣἤἥἦἧ\\
ἸἹἺἻἼἽἾἿ ἰ ἱ ἲ ἳ ἴ ἵ ἶ ἷ ὈὉὊὋὌὍ ὀὁὂὃὄὅ ὙὛὝὟ %
ὐὑὒὓὔὕὖὗ\\
ὨὩὪὫὬὭὮὯ ὠὡὢὣὤὥὦὧ ὰά ὲέ ὴή ὶί ὸό ὺύ ὼώ\\
ᾈᾉᾊᾋᾌᾍᾎᾏ ᾀᾁᾂᾃᾄᾅᾆᾇ ᾘᾙᾚᾛᾜᾝᾞᾟ ᾐᾑᾒᾓᾔᾕᾖᾗ\\
ᾨᾩᾪᾫᾬᾭᾮᾯ ᾠᾡᾢᾣᾤᾥᾦᾧ ᾸᾹᾺΆᾼ ᾰᾱᾲᾳᾴᾶᾷ\\
ῈΈ ῊΉῌ ῂῃῄῆῇ ῘῙῚΊ ῐ ῑ ΐ ῒ ῖ ῗ\\
ῠῡ ῢΰ ῤῥ ῦῧ ῨῩ ῪΎ Ῥ ῸΌῺΏῼ ῲῳῴῶῷ\\
ΆΈΉΊΌΎΏ ΪΫ άέήίόύώ ϊϋ ΐΰ %ϐ ϴϑ ϒϓϔ ϕϖϗ 
%Ϙϙ Ϛϛ Ϝϝ Ϟϟ Ϡϡ ϰ ϱ ϲ ϳ ϵ ϶· δ᾽\\

%Ϡ%ϡ ϰ ϱ ϲ %ϳ ϵ ϶· δ᾽

\textlatin{Greek Accents}\\
ʹ ͵ ͺ ; ΄ ΅ · ᾽ ι ᾽ ῀ ῁ ῍ ῎ ῏ ῝ ῞ ῟ ῭ ΅ ` ´ ῾ 

\hfill\today


\end{document}
