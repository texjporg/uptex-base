\documentclass{ujarticle}

%%%%%%%%
% set3 関連の制御をコマンドラインから行う
% ①,② のどれかを実行すればよい。
%   ① without set3
%   $ uplatex "\def\withsetthree{no}\input" jbib2-utf8.tex
%   ② with set3
%   $ uplatex jbib2-utf8.tex
%%%%%%
\def\withsetthreetmp{no}

\def\noop#1{}
\def\JLaTeX{J\LaTeX}
\def\JTeX{J\TeX}

\oddsidemargin0mm
\evensidemargin0mm
\topmargin-15mm
\textwidth162mm
\textheight245mm

\begin{document}
\section{欧文}

jbibtexのテスト\cite{article-minimal}。
jbibtexのテスト\cite{article-full}。
jbibtexのテスト\cite{article-crossref}。
jbibtexのテスト\cite{whole-journal}。
jbibtexのテスト\cite{inbook-minimal}。
jbibtexのテスト\cite{inbook-full}。
jbibtexのテスト\cite{inbook-crossref}。
jbibtexのテスト\cite{book-minimal}。
jbibtexのテスト\cite{book-full}。
jbibtexのテスト\cite{book-crossref}。
jbibtexのテスト\cite{whole-set}。
jbibtexのテスト\cite{booklet-minimal}。
jbibtexのテスト\cite{booklet-full}。
jbibtexのテスト\cite{incollection-minimal}。
jbibtexのテスト\cite{incollection-full}。
jbibtexのテスト\cite{incollection-crossref}。
jbibtexのテスト\cite{whole-collection}。
jbibtexのテスト\cite{manual-minimal}。
jbibtexのテスト\cite{manual-full}。
jbibtexのテスト\cite{mastersthesis-minimal}。
jbibtexのテスト\cite{mastersthesis-full}。
jbibtexのテスト\cite{misc-minimal}。
jbibtexのテスト\cite{misc-full}。
jbibtexのテスト\cite{inproceedings-minimal}。
jbibtexのテスト\cite{inproceedings-full}。
jbibtexのテスト\cite{inproceedings-crossref}。
jbibtexのテスト\cite{proceedings-minimal}。
jbibtexのテスト\cite{proceedings-full}。
jbibtexのテスト\cite{whole-proceedings}。
jbibtexのテスト\cite{phdthesis-minimal}。
jbibtexのテスト\cite{phdthesis-full}。
jbibtexのテスト\cite{techreport-minimal}。
jbibtexのテスト\cite{techreport-full}。
jbibtexのテスト\cite{unpublished-minimal}。
jbibtexのテスト\cite{unpublished-full}。
jbibtexのテスト\cite{random-note-crossref}。

\section{日本語のスタイルのテスト用のデータ群}

jbibtexのテスト\cite{inbook-full-j}。
jbibtexのテスト\cite{incol-full-j-1}。
jbibtexのテスト\cite{article-crossref-j}。
jbibtexのテスト\cite{article-crossref-jj}。
jbibtexのテスト\cite{whole-journal-j}。
jbibtexのテスト\cite{inbook-crossref-j}。
jbibtexのテスト\cite{whole-set-j}。
jbibtexのテスト\cite{inbook-crossref-j-1}。
jbibtexのテスト\cite{cvs}。
jbibtexのテスト\cite{ha}。
jbibtexのテスト\cite{costa}。
jbibtexのテスト\cite{mcclella}。
jbibtexのテスト\cite{dug}。
jbibtexのテスト\cite{sakawa}。
jbibtexのテスト\cite{ssl}。
jbibtexのテスト\cite{newman}。
jbibtexのテスト\cite{Rich}。
jbibtexのテスト\cite{goto}。
jbibtexのテスト\cite{磯崎}。
jbibtexのテスト\cite{斉藤}。
jbibtexのテスト\cite{sym}。
jbibtexのテスト\cite{eda}。
jbibtexのテスト\cite{dss}。
jbibtexのテスト\cite{cm}。
jbibtexのテスト\cite{reduce}。
jbibtexのテスト\cite{fp}。
jbibtexのテスト\cite{la}。
jbibtexのテスト\cite{あふれ無し}。
jbibtexのテスト\cite{ダム}。
jbibtexのテスト\cite{人名表記}。
jbibtexのテスト\cite{EUC日本語TeX}。
jbibtexのテスト\cite{multi}。
jbibtexのテスト\cite{marumoji}。
jbibtexのテスト\cite{maru}。

\section{yomi に平仮名を使う例}

jbibtexのテスト\cite{goto-h}。
jbibtexのテスト\cite{磯崎-h}。
jbibtexのテスト\cite{斉藤-h}。
jbibtexのテスト\cite{multi-h}。

\section{ソフトウェア科学会用のテストデータ}

jbibtexのテスト\cite{ama}。
jbibtexのテスト\cite{Arv}。
jbibtexのテスト\cite{cha}。
jbibtexのテスト\cite{dav}。
jbibtexのテスト\cite{den}。
jbibtexのテスト\cite{fis}。
jbibtexのテスト\cite{gajski}。
jbibtexのテスト\cite{suna1}。
jbibtexのテスト\cite{suna2}。
jbibtexのテスト\cite{toko1}。
jbibtexのテスト\cite{toko2}。
jbibtexのテスト\cite{suna86}。

\section{人工知能学会誌用テストデータ}

jbibtexのテスト\cite{onda}。
jbibtexのテスト\cite{近藤}。
jbibtexのテスト\cite{JSAI-1}。
jbibtexのテスト\cite{grosz}。

\section{改行位置のテスト}

jbibtexのテスト\cite{改行位置2a}。
jbibtexのテスト\cite{改行位置2b}。
jbibtexのテスト\cite{改行位置3a}。
jbibtexのテスト\cite{改行位置3b}。
jbibtexのテスト\cite{改行位置3c}。
% set3対応フォント+dviwareなら、「𠮷」もUTF-8で直接書ける。
\ifx\withsetthree\withsetthreetmp\else
jbibtexのテスト\cite{改行位置4a}。
jbibtexのテスト\cite{改行位置4b}。
jbibtexのテスト\cite{改行位置4c}。
jbibtexのテスト\cite{改行位置4d}。
\fi

\section{upTeX用テストデータ}
jbibtexのテスト\cite{森鷗外:百物語}。
jbibtexのテスト\cite{里見弴:極楽とんぼ}。
jbibtexのテスト\cite{国書:丿乀集}。
jbibtexのテスト\cite{グラハム:Unicode™標準入門}。
jbibtexのテスト\cite{test:misc0}。
jbibtexのテスト\cite{test:misc1}。
jbibtexのテスト\cite{test:misc2}。
% set3対応フォント+dviwareなら、「𠮷」もUTF-8で直接書ける。
\ifx\withsetthree\withsetthreetmp\else
jbibtexのテスト\cite{髙島𠮷野}。
\fi

%欧文8bit多バイトをタグに使うのはうまくいっていない。
%jbibtexのテスト\cite{Lautréamont}。
%jbibtexのテスト\cite{Schnitzler}。
%jbibtexのテスト\cite{Булгаков}。


%\bibliographystyle{jplain}
\bibliographystyle{jalpha}
%\bibliographystyle{jabbrv}
%\bibliographystyle{junsrt}
%\bibliographystyle{jname}

%\bibliographystyle{tipsj}%% 情報処理学会論文誌
%\bibliographystyle{jipsj}%% 情報処理学会欧文論文誌
%\bibliographystyle{tieice}%% 電子情報通信学会論文誌
%\bibliographystyle{jorsj}%% 日本オペレーションズリサーチ学会論文誌

\bibliography{jxampl,linebreak,jbtest}

\end{document}
